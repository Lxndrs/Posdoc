\subsection{Meteor location and trajectory}
\label{ssec:databases}

Data of the Caribbean meteoroid is available from the Geostationary Lightning Mapper (GLM) \citep{GOODMAN:2013} and the Center for Near Earth Object Studies (CNEOS) of the Jet Propultion Laboratory (JPL) of the NASA. We used the interactive database of both projects, available at \url{https://neo-bolide.ndc.nasa.gov/#/} and \url{https://cneos.jpl.nasa.gov/fireballs/}. From these databases, we found basic physical parameters such as event coordinates, altitude, total velocity asn its componentes, total radiated energy and calculated impact energy. These data are shown in table \ref{tab:Meteor-parameters}. 

\begin{table*}[t]
    \centering
    \caption{List of meteor basic parameters. Source: \url{https://cneos.jpl.nasa.gov/fireballs/}}
    \begin{tabular}{cc|c}
     \toprule
    Event date  (yyyy-mm-dd)&   & 2019-06-22 \\\hline
    Time (UT) & &21:25:48\\\hline
    Latitude (deg) & & 14.9\\\hline
    Longitude (deg) & & -66.2\\\hline
    Altitude (km) & &25.0\\\hline
    Velocity (km/s) & & 14.9\\\hline
    Duration (seconds)& & 4.873 \\\hline
    \multirow{3}{*}{Velocity components (km/s)}&$v_x$ & -13.4 \\\cline{2-3}
     & $v_y$& 6.0\\\cline{2-3}
     & $v_z$ & 2.5\\\hline
    Total radiated energy (J) & &\SI{294.7e10}{}\\\hline
    Calculated Total Impact energy (kt)&  & 6 \\
    \bottomrule
    \end{tabular}
    \label{tab:Meteor-parameters}
\end{table*}

Row 7 shows the velocity components of the bolide. $v_x$ is the velocity in the equatorial plane, positive towards the prime meridian, $v_z$ is directed towards the celestial north pole and $v_y$ completes the right-handed coordinate system, see figure \ref{fig:ref_velocity} in appendix \ref{app:velocity}. 

From GLM data we know that the bolide was visible for 4.873 seconds. The trajectory using the data is shown in figure  \ref{fig:bolide-trajectory}, among with the GLM records.

\begin{figure}
    \centering
    \includegraphics[width=\linewidth]{../figures/bolide-trajectory}
    \caption{In orange continuous line, the Caribbean bolide estimated trajectory using velocity components from table \ref{tab:Meteor-parameters}. In blue dots, GLM position records.}
    \label{fig:bolide-trajectory}
\end{figure}

 

  \section{RINEX data}
  \label{sec:GPS-data}
  We collected RINEX data for event from 38 stations that surrounded the impact coordinates for the day of the event. Since the event ocurred near 24:00 UT, we also obtained GPS data for the next day in case of there would be TIDs detected several hours after the fragmentation. We obtained TEC curves using the GOPI software and detrended the resulting time series for each receiver-satellite line of sight. %The spatial distribution of stations is shown in figure \ref{fig:stations_map}, and 
  Table \ref{tab:table-stations} shows the stations names, coordinates and proper citations.
  
  
%  \begin{figure}
%      \centering
%      \includegraphics[width=\linewidth]{./figures/caribbean_stations_map}
%      \caption{UNAVCO GPS stations map. The red star shows the positions where the bolide fragmented. The blue arrow shows the direction the bolide was moving in such instant. The green dots represent the positions of the stations we got GPS data.}
%      \label{fig:stations_map}
%  \end{figure}
  
   The obtained RINEX files are compressed in Hatanaka format, developed at the Geographical Survey Institute by Y. Hatanaka \citep{Kumar:2012}. From this files we may estimate the Slant Total Electron Content (sTEC) and the Vertical Total Electron Content (vTEC) which may be computed in the following way:

  The Total Electron content along the integrated path of the link $(s_i)$ at the frequency $f_i$ can be inferred from the phase delay $L_i$ of the frequency $f_i$ \citep{Emery:2017}:
  
  \begin{align}
    L_i = s_i - \frac{\SI{40.3082}{m^3.s^{-1}}}{f_i^2}sTEC_i
  \end{align}
  Combining two observations at two different frequencies $f_1$ and $f_2$ we may obtain two different phase delays $L_1$ and $L_2$ and derive the TEC along the signal path:

  \begin{align}
    sTEC = \frac{f_1^2f_2^2\left(L_1-L_2\right)}{\SI{40.3082}{m^3.s^{-1}}\left(f_1^2-f_2^2\right)}
  \end{align}

  In the other hand, the vertical Total Electron Content (vTEC) is computed from the sTEC as follows \citep{Kumar:2012}. In figure \ref{fig:TEC-sketch} we illustrate the satellite-receiver path and how the vTEC and sTEC are related.
  
  \begin{align}
    vTEC = \frac{sTEC-\left[b_R+b_S\right]}{S(\theta_I)}
  \end{align}

  where $b_R$ and $b_S$ are receiver and satellite biases, respectively. $\theta_I$ is the elevation angle in degrees, $S(\theta_I)$ is the obliquity factor with zenith angle $\psi$ at the Ionospheric Pierce Point (IPP):

  \begin{align}
    S(\theta_I) = \frac{1}{\cos\psi} = \left\lbrace 1-\frac{R_E\cos\theta_I}{R_E+h}\right\rbrace^{-1/2}
  \end{align}

  Where $R_E$ is the Earth radius in km and $h=\SI{350}{km}$ is the ionospheric shell above the earth's surface.

\begin{figure}
    \centering
    \includegraphics[width=\linewidth]{../figures/TEC_drawing.pdf}
    \caption{Schematic figure of TEC calculations in the ionosphere. The GPS satellite is located at an elevation of $\theta_I$ from the position of the receiver. The Ionospheric Pierce Point (IPP) is located at a height of $h=\SI{350}{km}$ above the Earth's surface. $\psi$ is the zenital angle. sTEC is measured along the satellite-receiver path and vTEC is measured in the radial direction sharing the same IPP as the satellite-receiver path.}
    \label{fig:TEC-sketch}
\end{figure}

  % Both parameters sTEC and vTEC are computed
 Using a software developed by Gopi K. Seemala, publicly available at \url{https://seemala.blogspot.com/}, we computed the slant TEC (sTEC) and vertical TEC (vTEC) for a several number of GPS satellites, each one identified with a PseudoRandom Noise code (PRN). An example of such TEC calculations is shown in figure \ref{fig:TEC-curve-example}. The behavior of the TEC curve is due to many factors, including the earth's rotation, solar activity, etc. TID's and wave-like features are not as prominent and are difficult to see. Since our sample is quite large, detrending process must be automatized in some degree, and it is critical to do this process adequately to have a correct interpretation of data \citep{Boris:2020}.
 
 \subsection{Detrending process}
 
 For detrending our data we used a method developed for detecting plasma bubbles in the equatorial region \citep{Pradipta:2015}, but proved to be effective for detecting Acoustic Gravity Waves (AGWs) and Traveling Ionospheric Disturbances (TIDs). 
 With this method we are able to infer the trend from our TEC data using a Savitsky-Golay filter. However, the results of this filtering are sensitive to the parameters we use for such filtering, thus altering significantly the quality of the detrended signal. 
 The TEC time series are assumed to be an additive combination of a signal and a trend \citep{Boris:2020}. Thus we tested the method of \citet{Pradipta:2015} over the sum of a known signal and a known trend and find the parameters of the Savitsky-Golay filter that recover in the best way the original signal. We used a test signal with the purpose to find out the most appropriate parameters for the filter and we show its functional form in appendix \ref{app:test-signal}. 


\begin{figure}
\centering
\includegraphics[width=\linewidth]{../figures/BOAV-vTEC-curves}
\label{fig:TEC-curve-example}
\caption{Example of vTEC curves obtained for GPS data from BOAV station.}
\end{figure}
 


On the other side, not necessarily all satellites could have detected ionospheric perturbations produced by TIDs, most of then should have detected other non related phenomena. To check which satellites have the best probabilities of detecting TIDs we compared in an azimuth-elevation map the satellites trajectory against the bolide trajectory. To do so we converted the bolide coordinates. The procedure is shown in \ref{app:azimuth}.

From Chelyabinsk event \citep{Yang:2014} we learned that high frequency TIDs (1.0 - 2.5 mHz) may travel at speed of $\sim 362 \pm 23~\mathrm{m~s^{-1}}$. Using this upper bound (since this event being less energetic, we can expect that similar TIDs in this case would travel at lower speed), we estimated the sphere where the TIDs should have propagated after a certain time interval. These propagation circles are shown in our azimuth-elevation maps.

From these figures that are representative of the trajectories observed for all stations we can see that the bolide is observed at an elevation of about 80 degrees and that the satellite that consistently was closest to the bolide trajectory is PRN 13, but after a few hours, PRN 2, PRN 17 and PRN 19 could intercept TIDs as they propagate from the bolide. 

\subsection{Morlet Wavelet}
\label{ssec:Morlet}
\begin{figure}
    \centering
    \includegraphics[width=\linewidth]{../figures/morlet.pdf}
    \caption{Example of Morlet wavelet. Since this is a complex function, we show in blue continous curve the real part and in green dashed curve the imaginary part. The Morlet wavelet consist in a sinosoid function with a determined frequency and damped to the left and right by a gaussian.}
    \label{fig:Morlet_wavelet}
\end{figure}

From our time series, we must distinguish some features that should appear if any TIDs has been detected. The TIDs are wave-like features constrained in a limited amount of time. A function that fits this description is the Morlet wavelet, which is given by \citep{Torrence:1998}:

\begin{align}
    \Psi_0(\eta) = \pi^{-1/4}\exp(i\omega_0\eta)\exp(-\eta^2/2)
\end{align}

Where $\omega_0$ is a non-dimensional frequency, usually set to be 6 to satisfy the admissibity condition \citep{Farge:1992} and $\eta$ is a non-dimensional "time parameter". The real and imaginary part of the Morlet wavelet is shown in figure \ref{fig:Morlet_wavelet}.

\subsection{TIDs velocity estimation}

%In figure \ref{fig:fragmentation-scheme} we analyze the scenario where TIDs are generated when the bolide is fragmented due to stagnation pressure. At this moment, a large amount of energy is released in the form of a shock wave that propagates isotropically. When the shock wave reaches back the ionosphere, the perturbed gas becomes detectable to our GPS sensors, if the perturbed area lies in the satellite-receiver Line Of Sight (LOS). In this case we should expect that TIDs appear in out time series in a certain time interval after the fragmentation occurred (the time needed for the shock wave reaching the ionosphere).

In figure \ref{fig:TID-scheme-2} we sketch the scenario where TIDs may be generated. This model explains that when the bolide enters the atmosphere perturbs the gas in its path. The bolide passes through the ionosphere before fragmentation then the TIDs generated should have been detected before fragmentation, too. The delay time in this case depends now in the bolide velocity as it penetrates the atmosphere.
%\begin{figure}
%    \centering
%\includegraphics[width=\linewidth]{figures/ionosphere_drawing.pdf}
%    \caption{Scheme of TIDs generation: here the bolide is fragmented due to stagnation pressure at height $h_B$. The fragmentation releases energy isotropically and eventually the blast reaches the ionosphere. The area of the perturbed ionosphere can or cannot be at the satellite-receiver Line of Sight (LOS).}
%    \label{fig:fragmentation-scheme}
%\end{figure}

\begin{figure}
    \centering
    \includegraphics[width=\linewidth]{../figures/ionosphere_drawing2.pdf}
       \caption{TIDs generation sketch. When the bolide passes through the atmosphere perturbs the gas in its way. The resulting TIDs can be detected before the bolide fragmentation.}
    \label{fig:TID-scheme-2}
\end{figure}