%% This is file `jasr-template.tex',
%% 
%% Copyright 2019-2020 Elsevier Ltd
%% 
%% This file is part of the 'Elsarticle Bundle'.
%% ---------------------------------------------
%% 
%% It may be distributed under the conditions of the LaTeX Project Public
%% License, either version 1.2 of this license or (at your option) any
%% later version.  The latest version of this license is in
%%    http://www.latex-project.org/lppl.txt
%% and version 1.2 or later is part of all distributions of LaTeX
%% version 1999/12/01 or later.
%% 
%% The list of all files belonging to the 'Elsarticle Bundle' is
%% given in the file `manifest.txt'.
%% 
%% Template article for Elsevier's document class `elsarticle'
%% with harvard style bibliographic references
%%
%% $Id: jasr-template.tex 188 2020-11-16 05:18:23Z rishi $
%%
%% Use the option review to obtain double line spacing
%\documentclass[times,review,preprint,authoryear]{elsarticle}

%% Use the options `twocolumn,final' to obtain the final layout
%% Use longtitle option to break abstract to multiple pages if overfull.
%% For Review pdf (With double line spacing)
%\documentclass[times,twocolumn,review]{elsarticle}
%% For abstracts longer than one page.
%\documentclass[times,twocolumn,review,longtitle]{elsarticle}
%% For Review pdf without preprint line
%\documentclass[times,twocolumn,review,nopreprintline]{elsarticle}
%% Final pdf
\documentclass[times,twocolumn,final,authoryear]{elsarticle}
%%
%\documentclass[times,twocolumn,final,longtitle]{elsarticle}
%%

%%
%% Stylefile to load JASR template
\usepackage{jasr}
\usepackage{framed,multirow}

%% The amssymb package provides various useful mathematical symbols
\usepackage{amssymb}
\usepackage{latexsym}

%% For line numbers
%\usepackage[switch]{lineno}

% Following three lines are needed for this document.
% If you are not loading colors or url, then these are
% not required.
\usepackage{url}
\usepackage{xcolor}
\usepackage{multirow}
%\usepackage{tablefootnote}
\definecolor{newcolor}{rgb}{.8,.349,.1}

\usepackage[citebordercolor=white]{hyperref}
\usepackage{rotating}
\usepackage{aas_macros}
\usepackage{pdflscape}
\usepackage{longtable}
\usepackage{amssymb,amsmath,amsthm}
\usepackage{siunitx}
\usepackage[utf8]{inputenc}
\usepackage{etoolbox}
\usepackage{fixltx2e}
\journal{Advances in Space Research}

\begin{document}

\verso{Tarango-Yong \textit{et. al}}

\begin{frontmatter}

\title{Ionospheric Disturbances in Mexican Territory Produced by Objects Entering the Atmosphere from Space \tnoteref{tnote1}}%
%\tnotetext[tnote1]{This is an example for title footnote coding.}

\author[1]{Jorge \snm{Tarango-Yong}\corref{cor1}}
\cortext[cor1]{Tel.: +52-443-476-5525;  \\
  email: jorge.tarango@comunidad.unam.mx
}

\author[1]{Mario \snm{Rodriguez-Martinez}\corref{cor2}}
\cortext[cor2]{email: m.rodriguez@enesmorelia.unam.mx}
%\fntext[fn1]{This is author footnote for second author.}
\author[1]{Raul \snm{Guti\'errez-Zalapa}\fnref{fn1}}
%% Third author's email
%\ead{author3@author.com}
%\author[2]{Given-name4 \snm{Surname4}}

\address[1]{Escuela Nacional de Estudios Superiores, UNAM, campus Morelia, Antigua Carretera a P\'atzcuaro No. 8701
Col. Ex Hacienda de San Jos\'e de la Huerta, Morelia, Michoac\'an, 58190, M\'exico}
%\address[2]{}

\received{1 May 2013}
\finalform{10 May 2013}
\accepted{13 May 2013}
\availableonline{15 May 2013}
\communicated{S. Sarkar}


\begin{abstract}
%%%
Please type your abstract here, and the rest of the text, figures,
tables, equations etc. in the main body. Please do not modify LaTeX\ 
commands unless you need to modify them and know how to do it.
%%%%
\end{abstract}

\begin{keyword}
%% MSC codes here, in the form: \MSC code \sep code
%% or \MSC[2008] code \sep code (2000 is the default)
%\MSC 41A05\sep 41A10\sep 65D05\sep 65D17
%% Keywords
\KWD Space Sciences\sep Atmosphere%\sep Keyword3
\end{keyword}

\end{frontmatter}

%% For linenumbers
%\linenumbers

%% main text
\section{Introduction}
\label{sec1}
Bolides impact the Earth every day, from the smallest ($d<\SI{0.5}{m}$), which are destroyed when entering the atmosphere to the biggest and most rare events, where an asteroid with a diameter $d>\SI{100}{m}$ is almost unaffected by the atmosphere and is technically unable to slow them down. If their diameter is greater than \SI{1}{km} the impact can be considered a global catastrophe \citep{Dudorov:2020}. Some famous modern cases are the Tunguska event 1n 1908 \citep{Wheeler:2019} and the Chelyabinsk event \citep{Yang:2014} in 2013.

Such events have been recently recorded in certain databases, such as the Center for Near Earth Objects Studies (CNEOS), Jet Propulsion Laboratory (JPL) and the Geostationary Lightning Mapper (GLM), publicly available at \url{https://cneos.jpl.nasa.gov/fireballs/} and \url{https://neo-bolide.ndc.nasa.gov/#/}, respectively.

We picked our interest in one certain event, that we nicknamed ``the Caribbean Meteoroid'', because this meteoroid is the most energetic which appear in both databases and thus is an interesting case for studying the potential Traveling Ionospheric Disturbances (TIDs) that could be produced. To do so, we collected RINEX data from nearby GPS stations from the UNAVCO network (also publicly available) and developed a method to detrend the resultant TEC series and estimate the TIDs propagation speed ... (The introduction also needs to be polished after getting the main results)


%Please use \verb+elsarticle.cls+ for typesetting your paper. Additionally,
%make sure not to remove the package \verb+jasr.sty+ already included in the
%preamble:
%\begin{verbatim} 
%  \usepackage{jasr}
%\end{verbatim}

%Make sure to have the file \verb+model5-names.bst+ to produce the references in
%the correct format. 

%Any instructions relevant to the \verb+elsarticle.cls+ are
%applicable here as well. See the online instruction available at:
%\makeatletter
%\if@twocolumn
%\begin{verbatim}
% https://support.stmdocs.in/wiki/
% index.php?title=Elsarticle.cls
%\end{verbatim}
%\else
%\begin{verbatim}
% https://support.stmdocs.in/wiki/index.php?title=Elsarticle.cls
% \end{verbatim}
%\fi
%\makeatother

%Following commands are defined for this journal which are not in
%\verb+elsarticle.cls+. 
%\begin{verbatim}
%  \received{}
%  \finalform{}
%  \accepted{}
%  \availableonline{}
%  \communicated{}
%\end{verbatim}


%\subsection{Subsection}
%This is only a \LaTeX\ template if you need one. 
%See the detailed guidelines for manuscript preparation 
%and submission at:
%\begin{verbatim}
%https://ees.elsevier.com/asr
%\end{verbatim}


\section{Observational data}
\label{sec:methodology}
\subsection{Meteors Database}

We selected a sample of meteors which were observed in mexican territory from the Geostationary Lightning Mapper \citep{GOODMAN:2013}. Orignally this project was designed to detect ligthning activity in earth's athmosphere, but has been proven that also can detect bolides entering the athmosphere. The detection comes from two satellites called GOES-16 and GOES-17 orbiting the earth in geostationary orbits. %Figure  shows the area where GOES-16 and GOES-17 detect activity on a lightning flash rate map; in mexican territory clearly both satellites can make detections. %Create a figure where the area coverage of GOES-16 and GOES-17 is shown.
We used the interactive database available at \url{https://neo-bolide.ndc.nasa.gov/#/}. These data are publicly available and easily downloaded from the same website. For each event we can obtain the recorded trajectory of meteors and the corresponding light curve. THe GLM satellites have an umbral magnitude for detection of -14. At this magnitude, a meteor is considered a bolide, and is expected to be at least decimeter-sized (in diameter) to reach such brightness. In the other hand, too bright meteors will saturate the detectors, and thus, lowering the quality of data. The result of this factors implies that the range in size of the objects in our sample varies in diameter between decimeter to meter size. %The sample was chosen following the next criteria:%selected in such way we chose the most probable objects to be detected by GPS sources in mexican territory and its surroundings: 
Each event also has assigned a confidence ratio, from low confidence to high, depending in how bright is the event itself and if the trajectory recorded by GLM ressembles (or not) a straight line. We chose only events whose confidence ratio is high, in oreder to be sure we chose the brightest objects, and thus, in the diameter size of bolides, we favored the meter-sized ones. In table \ref{tab:table-meteors} we list the object we chose to do this work, order in chronological order. The columns of the table, from left to right are and ID to enumerate the meteors in the sample, the date and time each meteor was detected, the duration of the detection, their respective coordinates and the estimated height of the meteor over the ground at the time of the detection. GOES-16 and GOES-17 systematically detect the meteors at slightly different positions and at slightly different times, so we calculated the mean of the duration, latitude and longitude reported by both satellites for each event, and used the standard deviation as the uncertainties.

From table \ref{tab:table-meteors} is also clear that the duration of all the bolides detection last less than a second. This obsevation suggests that the bolides remain undetected by the GLM satellites until they get fragmented due to stagnation presure when they release a huge amount of energy and thus they become detectable.

%From the light curves we can estimate the total radiated energy and then convert it to the total kinetic energy through a relation. With this energy we may estimate physical parameters with the cloud fragmentation model form wheeler et al.

%\begin{itemize}
%    \item The objects were detected inside mexican territory and its surroundings.
%    \item The objects were detected by both satellites GOES 16 and GOES 17 (stereo)
%    \item The detection has been assigned a high confidence ratio.
%\end{itemize}



\begin{table*}
  \centering
  \caption{List of meteors passing through Mexico. The events are listed in chronological order. The listed duration, latitude and longitude correspond to the mean of the measurements of both GOES satellites. The uncertainties correspond to the respecting mean deviation.}
\label{tab:table-meteors}
\begin{tabular}{rrrrrrr}
\hline
ID & Date of event & Start Time (UT)  & Duration (seconds) & Latitude (deg) & Longitude (deg) & Height (km)\\
\hline
01 & 2019-05-23 & 16:36:18 & $0.197\pm 0.0000$ & $24.30 \pm 0.000$ & $-101.60 \pm 0.849$  & 28\\
02 & 2019-07-18 & 14:30:30 & $0.058\pm 0.0000$ & $27.20 \pm 0.000$ & $-103.15 \pm 0.778$  & 72\\
03 & 2019-08-10 & 11:18:48 & $0.199\pm 0.0757$ & $21.50 \pm 0.000$ & $-102.50  \pm 0.849$ & 92\\
04 & 2019-10-03 & 07:55:33 & $0.106\pm 0.0297$ & $25.65 \pm 0.071$ & $-96.25 \pm   0.778$ & 74\\
05 & 2019-10-09 & 06:08:11 & $0.103\pm 0.0078$ & $23.60 \pm 0.000$ & $-111.95 \pm  0.212$ & 32\\
06 & 2019-11-16 & 09:36:04 & $0.396\pm 0.0134$ & $20.30 \pm 0.000$ & $-100.55 \pm  0.919$ & 82\\
07 & 2019-11-17 & 15:36:01 & $0.116\pm 0.0035$ & $31.70 \pm 0.000$ & $-117.70 \pm  1.131$ & 88\\
08 & 2019-11-19 & 07:57:40 & $0.097\pm 0.1138$ & $20.00 \pm 0.000$ & $-88.40 \pm  1.131$  & 99\\
09 & 2019-11-26 & 13:23:20 & $0.078\pm 0.0290$ & $23.90 \pm 0.000$ & $-108.70 \pm  0.849$ & 81\\
10 & 2019-12-04 & 09:42:54 & $0.173\pm 0.0028$ & $31.50 \pm 0.000$ & $-113.65 \pm  0.919$ & 77\\
11 & 2019-12-15 & 14:50:49 & $0.127\pm 0.0134$ & $27.70 \pm 0.000$ & $-114.10 \pm  0.849$ & 78\\
12 & 2019-12-29 & 16:16:35 & $0.062\pm 0.0134$ & $29.60 \pm 0.000$ & $-116.35 \pm  0.919$ & 79\\
13 & 2020-01-03 & 14:10:17 & $0.113\pm 0.0085$ & $30.20 \pm 0.000$ & $-117.65 \pm  0.919$ & 74\\
14 & 2020-01-06 & 16:39:27 & $0.118\pm 0.0042$ & $31.40 \pm 0.000$ & $-108.20 \pm  0.990$ & 81\\
15 & 2020-01-15 & 15:00:33 & $0.213\pm 0.1351$ & $19.45 \pm 0.071$ & $-95.55 \pm   0.919$ & 93\\
16 & 2020-02-12 & 09:25:40 & $0.210\pm 0.0226$ & $18.90 \pm 0.000$ & $-93.50 \pm   0.849$ & 90\\
17 & 2020-03-03 & 12:33:27 & $0.062\pm 0.0007$ & $18.25 \pm 0.071$ & $-106.35 \pm 0.636$  & 77\\
18 & 2020-03-31 & 19:31:52 & $0.105\pm 0.0573$ & $28.45 \pm 0.071$ & $-112.05 \pm  0.636$ & 61\\
19 & 2020-04-08 & 16:25:28 & $0.120\pm 0.0926$ & $26.10 \pm 0.000$ & $-93.90 \pm   0.849$ & 78\\
20 & 2020-04-18 & 17:43:25 & $0.139\pm 0.0106$ & $29.00 \pm 0.000$ & $-106.55 \pm  0.919$ & 82\\
21 & 2020-04-20 & 16:05:22 & $0.318\pm 0.1655$ & $28.15 \pm 0.071$ & $-97.85 \pm   1.061$ & 88\\
22 & 2020-04-25 & 11:03:09 & $0.323\pm 0.0813$ & $32.15 \pm 0.071$ & $-111.60 \pm  1.131$ & 84\\
23 & 2020-04-28 & 19:31:52 & $0.105\pm 0.0573$ & $28.45 \pm 0.071$ & $-112.05 \pm  0.636$ & 29\\
24 & 2020-05-08 & 10:06:16 & $0.490\pm 0.0750$ & $21.60 \pm 0.000$ & $-92.40 \pm   0.849$ & 81\\
25 & 2020-07-15 & 19:58:28 & $0.693\pm 0.0495$ & $24.00 \pm 0.000$ & $-108.35 \pm  0.495$ & 53\\
26 & 2020-08-07 & 13:29:57 & $0.163\pm 0.0057$ & $28.80 \pm 0.000$ & $-106.05 \pm  0.919$ & 89\\
27 & 2020-09-13 & 16:41:59 & $0.184\pm 0.0078$ & $28.45 \pm 0.071$ & $-113.75 \pm  0.919$ & 85\\
28 & 2020-09-30 & 12:28:11 & $0.100\pm 0.0078$ & $24.90 \pm 0.000$ & $-110.90 \pm  0.849$ & 83\\
29 & 2020-11-16 & 12:28:11 & $0.100\pm 0.0078$ & $24.90 \pm 0.000$ & $-110.90 \pm  0.849$ & 06\\
30 & 2020-11-17 & 12:53:41 & $0.404\pm 0.0262$ & $23.00 \pm 0.000$ & $-102.45 \pm  0.919$ & 93\\
31 & 2020-12-19 & 10:18:14 & $0.407\pm 0.0110$ & $21.95 \pm 0.071$ & $-101.60 \pm  0.990$ & 98\\
32 & 2020-12-23 & 09:43:01 & $0.148\pm 0.0014$ & $25.75 \pm 0.071$ & $-111.25 \pm  0.778$ & 81\\
33 & 2020-12-29 & 15:20:54 & $0.118\pm 0.0014$ & $16.80 \pm 0.000$ & $-102.20 \pm  0.707$ & 81\\
34 & 2021-03-31 & 09:01:17 & $0.753\pm 0.3083$ & $20.15 \pm 0.071$ & $-92.95 \pm  0.212$  & 24\\\hline
\end{tabular}
\end{table*}

\begin{figure}
  \centering
  \includegraphics[width=\linewidth]{../meteors_map}
  \caption{Positions of events from table \ref{tab:table-meteors}. The label of each point correspond to the ID (first column) of the referred table.}
  \label{fig:meteors-map}
\end{figure}

\subsection{GPS data}

This material is based on services provided by the GAGE Facility, operated by UNAVCO, Inc., with support from the National Science Foundation and the National Aeronautics and Space Administration under NSF Cooperative Agreement EAR-1724794.

We got RINEX data from 3 to 7 stations depending of the event location and data availability that surround the event place in all directions as possible. A list of the stations where we got RINEX data is available in table  . Most of the stations lie in mexican territory, but in some cases we required data from other stations to cover events near the mexican frontier at north or south.

\clearpage
\onecolumn
\begin{landscape}
%\begin{table*}
%    \centering
  \begin{longtable}{lllp{4cm}p{10cm}}
      \caption{List of GPS stations used for this work.}
      \label{tab:table-stations}
      \endfirsthead
      \endhead
    \hline
    Station name & Latitude & Longitude & Date of events  & Citation \\\hline
    BAR1\hyperlink{Hudnut}{${}^1$}\hyperlink{Hudnut2}{${}^5$}& 33.48 & -119.03 & 2019-12-29 2020-01-03 & UNAVCO Community, Hudnut, Kenneth, King, Nancy, Aspiotes, Aris G., Borsa, Adrian A., Determan, Daniel N., Galetzka, John E., Stark, Keith F., 2005, SCIGN-PBO Nucleus GPS Network - BAR1-Santa Barbara Island One P.S., The GAGE Facility operated by UNAVCO, Inc., GPS/GNSS Observations Dataset, \url{https://doi.org/10.7283/T5668BHN}.\\
    BLYT\hyperlink{Hudnut}{${}^1$} & 33.61 & -114.71 & 2019-12-29 2020-01-03 & Hudnut, Kenneth, King, Nancy, Aspiotes, Aris G., Borsa, Adrian A., Determan, Daniel N., Galetzka, John E., Stark, Keith F., 2006, SCIGN USGS GPS Network - BLYT-Blythe P.S., The GAGE Facility operated by UNAVCO, Inc., GPS/GNSS Observations Dataset, \url{https://doi.org/10.7283/T5HT2MKK}.\\
    CN23 & 17.26 & -88.78 & 2019-11-19 2020-01-15 2020-02-12& UNAVCO Community, 2012, COCONet GPS Network - CN23-BelmopanBZCR2012 P.S., The GAGE Facility operated by UNAVCO, Inc., GPS/GNSS Observations Dataset, \url{https://doi.org/10.7283/T5Q23XJH}.\\
    CN25 & 16.23 & -92.13 & 2020-01-15 & UNAVCO Community, 2014, COCONet GPS Network - CN25-ComitandDMEX2012 P.S., The GAGE Facility operated by UNAVCO, Inc., GPS/GNSS Observations Dataset, \url{https://doi.org/10.7283/T57W69G7}.\\
    GCFS & 19.31 & -81.18 & 2019-11-19& Watts, Anthony, 2016, COCONet GPS Network - GCFS-G\_CAYMAN\_CYM2014 P.S., The GAGE Facility operated by UNAVCO, Inc., GPS/GNSS Observations Dataset, \url{https://doi.org/10.7283/7ETV-X536}.\\
    GMPK\hyperlink{Hudnut}{${}^1$} & 33.05 & -114.83 & 2019-12-04 & UNAVCO Community, Hudnut, Kenneth, King, Nancy, Aspiotes, Aris G., Borsa, Adrian A., Determan, Daniel N., Galetzka, John E., Stark, Keith F., 2005, SCIGN-PBO Nucleus GPS Network - GMPK-Glamis Peak P.S., The GAGE Facility operated by UNAVCO, Inc., GPS/GNSS Observations Dataset, \url{https://doi.org/10.7283/WCHN-H687}.\\
    GUAT\hyperlink{Garnier}{${}^2$} & 14.59 & -90.52 & 2020-02-12 & DeMets, Charles, Cosenza-Muralles, Beatriz, 2021, Central America 2018 - Guatemala, The GAGE Facility operated by UNAVCO, Inc., GPS/GNSS Observations Dataset, \url{https://doi.org/10.7283/KH2R-K704}.\\
    GUAX\hyperlink{Hudnut}{${}^1$} & 28.88 & -118.29 & 2019-10-09 2019-12-15 2019-12-29 2020-01-03 2020-03-31 2020-07-15 2020-09-13 2020-09-30 2020-12-23 & Hudnut, Kenneth, King, Nancy, Aspiotes, Aris G., Borsa, Adrian A., Determan, Daniel N., Galetzka, John E., Stark, Keith F., 2001, SCIGN USGS GPS Network - GUAX-Isla Guadalupe P.S., The GAGE Facility operated by UNAVCO, Inc., GPS/GNSS Observations Dataset, \url{https://doi.org/10.7283/T5GX48T2}.\\
    IAGX & 29.03 & -113.17 & 2019-12-04 & Gonzalez-Ortega, Alejandro, Galetzka, John E., Gonzalez, Javier, 2018, CICESE REGNOM GPS Network - IAGX-iagxREGNOMmx2018 P.S., The GAGE Facility operated by UNAVCO, Inc., GPS/GNSS Observations Dataset, \url{https://doi.org/10.7283/DGWN-A627}.\\
    INEG & 21.85 & -102.28 & 2020-07-15 2020-08-07 2020-09-30 2020-11-16 2020-11-17 2020-12-19 & No citations were found\\
    KVTX & 27.55 & -97.89 & 2019-05-23 2019-07-18 2019-08-10 2019-10-03 2019-11-17 2020-04-08 2020-04-18 2020-04-20 2020-05-08 2020-08-07 & UNAVCO Community, 2007, PBO GPS Network - KVTX-KingsvilleTX2006 P.S., The GAGE Facility operated by UNAVCO, Inc., GPS/GNSS Observations Dataset, \url{https://doi.org/10.7283/T5J38QH8}.\\
    MDO1 & 30.68 & -104.02 & 2019-07-18& No citations were found\\
    MGO5 & 30.68 & -104.02 & 2020-04-20 2020-08-07 & No citations were found\\
    MGW3 & 29.62 & -89.95 & 2020-04-08 2020-04-20 2020-05-08 & No citations were found\\
    OXTH & 16.29 & -95.24 & 2020-01-15 2020-02-12 & DeMets, Charles, Cabral-Cano, Enrique, 2008, Oaxaca GPS Network - OXTH-Tehuantepec P.S., The GAGE Facility operated by UNAVCO, Inc., GPS/GNSS Observations Dataset, \url{https://doi.org/10.7283/T5Q81B5V}.\\
    OXUM\hyperlink{Graham}{${}^3$} & 15.66 & -96.50 & 2021-03-31 & Cabral-Cano, Enrique, Salazar-Tlaczani, Luis, 2015, TLALOCNet - OXUM-oxum\_tnet\_mx2001 P.S., The GAGE Facility operated by UNAVCO, Inc., GPS/GNSS Observations Dataset, \url{https://doi.org/10.7283/T5J964RP}.\\
    P001 & 31.95 & -112.80 & 2020-04 25 & UNAVCO Community, 2008, PBO GPS Network - P001-Organ\_PipeAZ2007 P.S., The GAGE Facility operated by UNAVCO, Inc., GPS/GNSS Observations Dataset, \url{https://doi.org/10.7283/T5DR2SGP}.\\
    P014 & 31.97 & -11.09 & 2019-12-04 2019-12-29 2020-01-03 2020-01-06 2020-04-25 & UNAVCO Community, 2008, PBO GPS Network - P014-Sahuarita\_AZ2007 P.S., The GAGE Facility operated by UNAVCO, Inc., GPS/GNSS Observations Dataset, \url{https://doi.org/10.7283/T5DJ5CMK}.\\
    P807 & 30.49 & -98.82 & 2019-11-17 2020-01-06 2020-04-20 2020-11-17 & UNAVCO Community, 2012, PBO GPS Network - P807-EcRockStPkTX2012 P.S., The GAGE Facility operated by UNAVCO, Inc., GPS/GNSS Observations Dataset, \url{https://doi.org/10.7283/T5TQ5ZKM}.\\
    PLPX & 31.59 & -115.15 & 2019-12-04 & UNAVCO Community, 2011, PBO GPS Network - PLPX-Las\_PintasMX2010 P.S., The GAGE Facility operated by UNAVCO, Inc., GPS/GNSS Observations Dataset, \url{https://doi.org/10.7283/T5K64G3T}.\\
    PTEX & 32.29 & -116.52 & 2019-12-29 2020-01-03 2020-09-13 2020-12-23 & UNAVCO Community, 2011, PBO GPS Network - PTEX-Testerazo\_MX2011 P.S., The GAGE Facility operated by UNAVCO, Inc., GPS/GNSS Observations Dataset, \url{https://doi.org/10.7283/T5610XBP}.\\
    RG06 & 32.63 & -107.86 & 2020-04-25 & Sheehan, Anne, 2007, Rio Grande Rift GPS Network - RG06-RG06FaywodNM2006 P.S., The GAGE Facility operated by UNAVCO, Inc., GPS/GNSS Observations Dataset, \url{https://doi.org/10.7283/T5668BFR}.\\
    RG07 & 32.50 & -106.84 & 2020-01-06 & Sheehan, Anne, 2007, Rio Grande Rift GPS Network - RG07-RG07CrucesNM2006 P.S., The GAGE Facility operated by UNAVCO, Inc., GPS/GNSS Observations Dataset, \url{https://doi.org/10.7283/T5KD1W45}. \\
    SG33 & 31.77 & -106.51 & 2019-11-17 2020-04-18 2020-08-07 & Harder, Steven, Kaip, Galen, Montana, Carlos, 2004, SuomiNet-G GPS Network - SG33-UTEP P.S., The GAGE Facility operated by UNAVCO, Inc., GPS/GNSS Observations Dataset, \url{https://doi.org/10.7283/T50863KQ}.\\
    TGMX  & 20.87 & -86.87 & 2021-03-31 & UNAVCO Community, 2015, COCONet GPS Network - TGMX-PtoMor\_TG\_MX2015 P.S., The GAGE Facility operated by UNAVCO, Inc., GPS/GNSS Observations Dataset, \url{https://doi.org/10.7283/T5154FB7}.\\
    TNAM & 20.54 & -103.97 & 2020-03-03 2020-07-15 2020-09-30 2020-11-16 2020-11-17 2020-12-19 & UNAVCO Community, 2014, TLALOCNet - TNAM-TNAM\_TNET\_MX2014 P.S., The GAGE Facility operated by UNAVCO, Inc., GPS/GNSS Observations Dataset, \url{https://doi.org/10.7283/T5QF8R4R}.\\
    TNAT & 18.13 & -98.04 & 2020-01-15 & UNAVCO Community, 2014, TLALOCNet - TNAT-TNAT\_TNET\_MX2014 P.S., The GAGE Facility operated by UNAVCO, Inc., GPS/GNSS Observations Dataset, \url{https://doi.org/10.7283/T5G15Z4S}.\\
    TNBA & 28.97 & -113.55 & 2019-10-09 2019-11-26 2019-12-15 2019-12-29 2020-01-03& UNAVCO Community, 2015, TLALOCNet - TNBA-TNBA\_TNET\_MX2014 P.S., The GAGE Facility operated by UNAVCO, Inc., GPS/GNSS Observations Dataset, \url{https://doi.org/10.7283/T57M0688}.\\
    TNCC & 18.79 & -103.17 & 2020-03-03 & UNAVCO Community, 2015, TLALOCNet - TNCC-TNCC\_TNET\_MX2015 P.S., The GAGE Facility operated by UNAVCO, Inc., GPS/GNSS Observations Dataset, \url{https://doi.org/10.7283/T50R9MSK}.\\
    TNCM & 19.50 & -105.04 & 2020-03-03 2020-04-28 & UNAVCO Community, 2014, TLALOCNet - TNCM-TNCM\_TNET\_MX2014 P.S., The GAGE Facility operated by UNAVCO, Inc., GPS/GNSS Observations Dataset, \url{https://doi.org/10.7283/T5B856FW}.\\
    TNCN & 18.55 & -101.97 & 2020-11-16 2020-12-29 & UNAVCO Community, 2016, TLALOCNet - TNCN-TNCN\_TNET\_MX2016 P.S., The GAGE Facility operated by UNAVCO, Inc., GPS/GNSS Observations Dataset, \url{https://doi.org/10.7283/T5610XQM}.\\
    TNCU & 28.45 & -106.79 & 2019-05-23 2019-07-18 2019-08-10 2019-11-17 2019-12-15 2020-01-06 2020-03-31 2020-04-18 2020-07-15 2020-08-07 2020-11-17 2020-12-19 & UNAVCO Community, 2014, TLALOCNet - TNCU-CuauhtemocTN2014 P.S., The GAGE Facility operated by UNAVCO, Inc., GPS/GNSS Observations Dataset, \url{https://doi.org/10.7283/T5V69GV2}.\\
    TNGF & 19.33 & -99.18 & 2020-11-16 2020-12-29 & Cabral-Cano, Enrique, Salazar-Tlaczani, Luis, 2016, TLALOCNet GPS Network - TNGF\_Geofisica-UNAM\_Mexico\_City\_TNET\_mx2015 P.S., The GAGE Facility operated by UNAVCO, Inc., GPS/GNSS Observations Dataset, \url{https://doi.org/10.7283/T53X851M}.\\
    TNHM & 29.08 & -110.97 & 2019-10-09 2019-11-26 2019-12-04 2019-12-15 2019-12-29 2020-01-03 2020-03-31 2020-04-18 2020-07-15 2020-08-07 2020-09-13 2020-09-30 2020-12-23 & UNAVCO Community, 2014, TLALOCNet - TNHM-hermosilloTN2014 P.S., The GAGE Facility operated by UNAVCO, Inc., GPS/GNSS Observations Dataset, \url{https://doi.org/10.7283/T5KP80FV}.\\
    TNMS & 20.53 & -104.80 & 2019-10-09 2019-11-26 2019-12-15 2020-03-03 2020-07-15 & UNAVCO Community, 2014, TLALOCNet - TNMS-TNMS\_TNET\_MX2014 P.S., The GAGE Facility operated by UNAVCO, Inc., GPS/GNSS Observations Dataset, \url{https://doi.org/10.7283/T56H4FQ5}.\\
    TNNP & 16.12 & -97.14 & 2020-04-28 & Cabral-Cano, Enrique, Salazar-Tlaczani, Luis, DeMets, Charles, 2016, TLALOCNet - TNNP-tnnp\_tnet\_mx2015 P.S., The GAGE Facility operated by UNAVCO, Inc., GPS/GNSS Observations Dataset, \url{https://doi.org/10.7283/T5N29V96}.\\
    TNNX & 17.41 & -97.22 & 2020-01-15 2020-02-12 2020-12-29 2021-03-31 & UNAVCO Community, 2014, TLALOCNet - TNNX-TNNX\_TNET\_MX2014 P.S., The GAGE Facility operated by UNAVCO, Inc., GPS/GNSS Observations Dataset, \url{https://doi.org/10.7283/T52R3PZ0}.\\
    TNPP & 31.34 & -113.63 & 2019-12-04 2020-03-31 2020-04-25 & UNAVCO Community, 2015, TLALOCNet - TNPP-TNPP\_TNET\_MX2015 P.S., The GAGE Facility operated by UNAVCO, Inc., GPS/GNSS Observations Dataset, \url{https://doi.org/10.7283/T5CC0Z0M}.\\
    TNSJ & 16.17 & -96.49 & 2020-12-29 & UNAVCO Community, 2016, TLALOCNet - TNSJ-tnsj\_tnet\_mx2015 P.S., The GAGE Facility operated by UNAVCO, Inc., GPS/GNSS Observations Dataset, \url{https://doi.org/10.7283/T59S1PF1}.\\
    TSFX & 30.93 & -114.81 & 2020-09-13 2020-12-23 & Gonzalez-Ortega, Alejandro, Galetzka, John E., Gonzalez, Javier, 2018, CICESE REGNOM GPS Network - TSFX-tsfxREGNOMmx2016 P.S., The GAGE Facility operated by UNAVCO, Inc., GPS/GNSS Observations Dataset, \url{https://doi.org/10.7283/AGEA-2G27}.\\
    UAGU & 21.92 & -102.32 & 2019-05-23 2019-07-18 2019-08-10 2019-10-03 2019-11-17 2019-11-26 2019-12-15 2020-04-18& Cabral-Cano, Enrique, Salazar-Tlaczani, Luis, 2015, TLALOCNet - UAGU-uagu\_tnet\_mx2008 P.S., The GAGE Facility operated by UNAVCO, Inc., GPS/GNSS Observations Dataset, \url{https://doi.org/10.7283/T5513WK7}.\\
    UCOE\hyperlink{Graham}{${}^3$} & 19.81 & -101.69 & 2019-08-10 2020-11-16 2020-11-17 2020-12-19 & Cabral-Cano, Enrique, Salazar-Tlaczani, Luis, 2015, TLALOCNet - UCOE-ucoe\_tnet\_mx2003 P.S., The GAGE Facility operated by UNAVCO, Inc., GPS/GNSS Observations Dataset, \url{https://doi.org/10.7283/T51834VW}.\\
    UGEO\hyperlink{Marquez}{${}^4$} & 20.69 & -103.35 & 2019-08-10& Marquez-Azua, Bertha, DeMets, Charles, Cabral-Cano, Enrique, Salazar-Tlaczani, Luis, 2015, TLALOCNet - UGEO-ugeo\_tnet\_mx1998 P.S., The GAGE Facility operated by UNAVCO, Inc., GPS/GNSS Observations Dataset, \url{https://doi.org/10.7283/T58S4N9N}.\\
    UHSL & 29.57 & -95.65 & 2020-04-08 & Wang, Guoquan, 2014, HoustonNet GPS Network - UHSL-SugarLandUSA2014 P.S., The GAGE Facility operated by UNAVCO, Inc., GPS/GNSS Observations Dataset, \url{https://doi.org/10.7283/T55X271S}.\\
    UHWL & 30.06 & -94.98 & 2020-12-19 & Wang, Guoquan, 2014, HoustonNet GPS Network - UHWL-West Liberty Airport(Deep) P.S., The GAGE Facility operated by UNAVCO, Inc., GPS/GNSS Observations Dataset, \url{https://doi.org/10.7283/T53R0R5P}.\\
    UNPM & 20.86 & -86.86 & 2019-11-19 2020-01-15 2020-02-12 2020-05-08 & UNAVCO Community, 2012, COCONet GPS Network - UNPM-Puerto\_Morelos\_MX\_2007 P.S., The GAGE Facility operated by UNAVCO, Inc., GPS/GNSS Observations Dataset, \url{https://doi.org/10.7283/J1GD-5S40}.\\
    USMX & 29.82 & -109.68 & 2019-12-29 2020-01-03 2020-01-06 2020-04-25 2020-08-07 2020-09-30& Bennett, Rick, 2004, Northwest Mexico GPS Network - USMX-Universidad de la Sierra P.S., The GAGE Facility operated by UNAVCO, Inc., GPS/GNSS Observations Dataset, \url{https://doi.org/10.7283/T5W957CQ}.\\
    UXAL\hyperlink{Graham}{${}^3$} & 19.52 & -96.92 & 2019-10-03 2020-01-15 2020-02-12 2020-04-08 2020-05-08 2020-12-19 2021-03-31 & Cabral-Cano, Enrique, Salazar-Tlaczani, Luis, 2015, TLALOCNet - UXAL-uxal\_tnet\_mx2005 P.S., The GAGE Facility operated by UNAVCO, Inc., GPS/GNSS Observations Dataset, \url{https://doi.org/10.7283/T5DJ5D1C}.\\
    WEPD & 29.69 & -95.23 & 2020-04-20 & Wang, Guoquan, 2014, HoustonNet GPS Network - WEPD-willmselementary P.S., The GAGE Facility operated by UNAVCO, Inc., GPS/GNSS Observations Dataset, \url{https://doi.org/10.7283/T5NZ85RB}.\\
    WMOK & 34.74 & -98.78 & 2020-04-20 & UNAVCO Community, 2005, PBO GPS Network - WMOK-WichitaMtnOK2005 P.S., The GAGE Facility operated by UNAVCO, Inc., GPS/GNSS Observations Dataset, \url{https://doi.org/10.7283/T59021Q6}.\\
    WWMT\hyperlink{Hudnut}{${}^1$} & 33.96 & -116.65 & 2019-12-29 2020-01-03& Hudnut, Kenneth, King, Nancy, Aspiotes, Aris G., Borsa, Adrian A., Determan, Daniel N., Galetzka, John E., Stark, Keith F., 2006, SCIGN USGS GPS Network - WWMT-Whitewater Mountain P.S., The GAGE Facility operated by UNAVCO, Inc., GPS/GNSS Observations Dataset, \url{https://doi.org/10.7283/T5H993F2}.\\
    YESX & 28.38 & -108.92 & 2019-11-26 2019-12-15 2020-01-06 2020-04-18 2020-04-28 2020-07-15& Bennett, Rick, 2004, Northwest Mexico GPS Network - YESX-Yecora P.S., The GAGE Facility operated by UNAVCO, Inc., GPS/GNSS Observations Dataset, \url{https://doi.org/10.7283/T5RJ4GPF}.\\\hline
    % \end{table*}
  \end{longtable}
    \begin{minipage}{0.9\linewidth}
      \footnotesize
      Related articles:
      
      \hypertarget{Hudnut}{${}^1$}\citet{Hudnut:2002},
      %
      \hypertarget{Garnier}{${}^2$}\citet{Garnier:2021}, 
      %
      \hypertarget{Graham}{${}^3$}\citet{Graham:2016}
      
      \hypertarget{Marquez}{${}^4$}\href{https://doi.org/10.7283/T58S4N9N}{B. Marquez-Azua, E. Cabral-Cano, F. Correa-Mora and C. DeMets, 2004. A model for Mexican neotectonics based on Nationwide GPS measurements, 1993-2001, Geofisica Internacional, v. 43, p.319-330}
      
      \hypertarget{Hudnut2}{${}^5$}\href{https://doi.org/10.7283/T5668BHN}{Hudnut, K. W., Y. Bock, J. E. Galetzka, F. H. Webb, and W. H. Young, The Southern California Integrated GPS Network (SCIGN), Proceedings of the International Workshop on Seismotectonics at the Subduction Zone, Y. Fujinawa (ed.), NIED, Tsukuba, Japan, pp. 175-196, 1999}
    
    
    
    \end{minipage}
  \end{landscape}
  \clearpage
  \twocolumn
%For the selected sample, we obtained RINEX data from the TlalocNet \citep{Cabral-Cano:2018} and UNAVCO network databases to study potential alterations in the ionospehre due to the presence of the passing meteor at the day the meteor was reported. For each event, we downloaded data from stations that surrounds the place where the event was detected (usually 3 to 5 stations. )The list of the sample meteors is shown in table \ref{tab:table-meteors}. The events are in chronological order. The reported duration, latitude and longitude correspond to the mean between measurements from satellites GOES-16 and GOES 17; in the same way, the uncertainties correspond to the standard deviation. Also their respective positions are available in figure \ref{fig:meteors-map}, where each label correspond to the ID (first column) of table \ref{tab:table-meteors}.
     
%Using the data provided by TlalocNet and UNAVCO, we procceeded to obtain TEC parameters with the GPS\_GOPI software, available at \url{https://seemala.blogspot.com/}. This software takes as input the RINEX data (the navigation file is no strictly neccesary), and the outuput consists in in the vTEC and sTEC measurements for the PRNs of the whole day the event occurred, as well as the averaged TEC as function of time. We obtained vTEC maps for the events in table (\ref{tab:table-meteors}) and their respective next and previous day.
     
%Final idea: use GOPI software to get the vTEC data from the day of each event and the previous days.


\section{Bolides physical parameters}
\label{sec:bolides}
Enter Raul's work here
\section{Ionospheric background and vTEC maps}
\label{sec:vTEC-maps}
     
Ionospheric perturbations also can take place due to space weather and geomagnetic storms. So, in order to discard such events we investigated the space weather in the day each event occured. In figure (name) we present the geomagnetic \textit{Kp} index for some events. We discarded events whose Kp index is equal or grater than 4 in the day of the event or shortly before. Also we present in figure (name) the vTEC perturbation maps for the same events in a three day series, centered in the event date. The estimated meteor trajectory, obtained from the GLM data is presented in black, continous line, while the linear fit to the GOES-16 and GOES-17 data are presented with the red dashed line, and work as boudary errors.
     
     
     
%\section{Frecuency Analysis}
     
\section{Discussion}
\label{sec:discission}

We collected data for a sample of GPS station from UNAVCO network around the Caribbean bolide was detected in the same day and next day. We used data from the GLM, where the total energy was estimated from light curves, and USG sensors, where energy data is available to spot the bolide position and estimate the trajectory at the fragmentation instant. 
For GPS data, we adapted the method from \citet{Pradipta:2015} for detrending all the resultant TEC curves to remove the effects of Earth rotation and solar activity and make the wave-like features more evident. In order to discriminate which stations made a detection from the ones who didn't, we compared the detrended time series with a Morlet wavelet. We discarded other sources of GWs such like slow solar wind reached by fast wind, fluctuations in solar X-rays, coronal mass ejections and solar terminator effects. We found TIDs features in sTEC series from stations CN04, CN40, GRE1, TTSF, TTUW, BOAV, KOUG and KOUR stations with at least one satellite. 

%\section{How to include a figure?}
%Change the file name of the figure below with your own 
%figure. And remember to change the figure caption too!
%\begin{verbatim}
%\begin{figure}
%  \centering
%  \includegraphics[scale=0.5]{fig01}
%  \caption{Write the figure caption here.}
%  \label{fig:pendulum}
%\end{figure}
%\end{verbatim}
%\begin{figure}
%\centering
%\includegraphics[scale=0.5]{fig01.pdf}
%\caption{Write the figure caption here.}
%\label{fig:pendulum}
%\end{figure}

%\subsection{And a table?}
%Just replace the text/values in the template table below 
%with your own. You can change the number of 
%lines/rows as necessary.
   
%\begin{table*}
%\centering
%\caption{Title of the table should be at the top}
%\begin{tabular}{|l|l|l|l|}
%\hline
%Column Name1    & Column Name2 & Column Name3 & Column Name4 \\
%\hline
%Parameter Name1 & Value        & Value        & Value        \\
%\hline
%Parameter Name2 & Value        & Value        & Value        \\
%\hline
%Parameter Name3 & Value        & Value        & Value        \\
%\hline
%\end{tabular}
%\end{table*}

%\subsection{Equations}
%Conventionally, in mathematical equations, variables and
%anything that represents a value appear in italics. 
%All equations should be numbered for easy referencing. 
%The number should appear at the right margin.
%\begin{eqnarray}
%S'_{\mathrm{pg}} = \frac{S_{\mathrm{pg}} - \mathrm{min}(S_{\mathrm{pG}})}
%  {\mathrm{max}(S_{\mathrm{pG}} - \mathrm{min}(S_{\mathrm{pG}}))}
%\end{eqnarray}
%In mathematical expressions 
%in running text "/" should be used
%for division (not a horizontal line).

%\section{Citations}
%Citations in the text can be made using\\[6pt]
%\verb+\citet{NewmanGirvan2004}+\\[6pt]
%for citation in running text like in 
%\citet{NewmanGirvan2004} or using\\[6pt]
%\verb+\citep{Vehlowetal2013,NewmanGirvan2004}+\\[6pt]
%for citation within parentheses like in 
%\citep{Vehlowetal2013,NewmanGirvan2004}.

%Please use the actual \verb+\cite+ command in the text.
%Also, please double-check the \verb+\citep+ command.

%\section{Reference style}
%You can include the references in the main text file in \LaTeX
%format. Alternately, you can include a separate bibliography
%file (refs.bib in this example) and run the following set of 
%commands:
%\begin{verbatim}
%==================

%pdflatex myfile.tex

%bibtex myfile (no extension in this line!)

%pdflatex myfile.tex

%pdflatex myfile.tex

%==================
%\end{verbatim}

%\section{A sample entry in the bibliography file}
%{\fontsize{7.5pt}{9.6pt}\selectfont
%\begin{verbatim}
%==================

%@ARTICLE{NewmanGirvan2004,
%  author  = {Newman, M. E. J. and Girvan, M.},
%  title   = {Finding and evaluating community 
%               structure in networks},
%  journal = {Phys. Rev. E.},
%  volume  = {69},
%  number  = {21},
%  year    = {2004},
%  pages   = {026113}
%}

%==================
%\end{verbatim}
%}

\section{Acknowledgments}
\label{sec:acknowledgments}
\input{../Acknowledgments}
%% Bibliography
%% Author year style
\appendix
% \appendixpage
% \addappheadtotoc

% This is the appendix, where we may insert detailed mathematical procedures or other data which may be useful but also will make the main
% text too large so this should be consulted only in case of necessity
\section{Bolide velocity components}
\label{app:velocity}
The USG database from JPL is a great source of information about the brightest bolides that have entered the atmosphere since 1988. However, the velocity components $(v_x, v_y, v_z)$ are displayed in a geocentric Earth-fixed reference frame explained as follows:

$v_x$ lies in the Earth's equatorial plane, parallel to the equator and points towards the prime meridian, $v_z$ is parallel to the earth's rotational axis and points towards the north celestial pole and $v_y$ completes the right-handed reference system. Figure  shows graphically these velocity components.

In the other hand, it might be useful and more intuitive to describe the velocity components in terms of the local reference system composed by $(v_{lon}, v_{lat}, v_h)$, where $v_{lon}$ is the velocity component parallel to the equatorial plane, positive towards east, $v_{lat}$ is parallel to the polar plane, positive to north, and $v_h$ is the radial component pointing towards the center of Earth. Figure \ref{fig:ref_velocity} sketches both reference systems simultaneously.

\begin{figure}
    \centering
    \includegraphics[width=\linewidth]{../figures/meteor_velocity_references.pdf}
    \caption{Velocity components of bolide. The geocentric Earth fixed reference system is sketched in black and the local spherical reference system in blue. The components parallel to the equatorial plane are not shown since are perpendicular to the plane of the page.}
    \label{fig:ref_velocity}
\end{figure}

The transformation between both reference systems in the north-west hemisphere is given as follows:

\begin{align}
v_{lon} &= v_x \label{eq:vlon}\\
v_{lat} &= v_z\cos{L} - v_y\sin{L} \label{eq:vlat}\\
v_h     &= v_y\cos{L} + v_z\sin{L} \label{eq:vh}
\end{align}

Where $L$ is the bolide latitude. With this information, we can estimate the velocity tangential to the Earth's surface as $v_{tan} = \left((v_{lon})^2 + (v_{lat})^2\right)^{1/2}$ and finally the trajectory angle respect to the normal as $\tan\theta = \frac{v_{tan}}{v_h}$. Using equations (\ref{eq:vlon}) to (\ref{eq:vh}) we obtain the following:

\begin{align}
    v_{tan} &= \left((v_x)^2 + (v_z\cos{L}-v_y\sin{L})^2\right)^{1/2} \\
    \tan\theta &= \frac{\left((v_x)^2 + (v_z\cos{L}-v_y\sin{L})^2\right)^{1/2}}{v_y\cos{L} + v_z\sin{L}}  \label{eq:meteor-angle}
\end{align}

Substituting the corresponding data from table \ref{tab:Meteor-parameters} into equation (\ref{eq:meteor-angle}) we find that that the trajectory angle of the meteor respect to the normal is $\theta \approx 64.4^{\circ}$

\section{Detrending test signal}
\label{app:test-signal}
Figure \ref{fig:test-signal} shows the form of this test time series, described mathematically as follows \citep{Boris:2020}:
 \begin{align}
     I_R(t) &= A\cdot \exp\left[-0.5\left(\frac{t-t_m}{d_t}\right)^2\right]\cdot \sum^{i}_{n}\sin\left(\omega_i t\right) \label{eq:ref-signal}\\
     Trend(t) &= B\cdot \left|t - t_0\right|^3 \label{eq:trend}
 \end{align}
 
 \begin{figure*}
     \centering
     \includegraphics[width=\linewidth]{../figures/test_signal}
     \caption{Initial setup for testing detrending method. a) Reference signal given by equation (\ref{eq:ref-signal}). b) Superposition between a) and the trend given by (\ref{eq:trend})}.
     \label{fig:test-signal}
 \end{figure*}
 
 Where equation (\ref{eq:ref-signal}) is the signal we want to detrend and equation (\ref{eq:trend}) is the trend we want to remove. $A$ is the amplitude of the signal, set as $\SI{0.2}{TECU}$, $t_m$ is a parameter which determines the position of the envelope maximum, set as $\SI{250}{min}$, which is the half of the array length, $d_t$ is the half width of the envelope, set as $\SI{50}{min}$ and $\omega_i$ are the frequencies of three harmonics with periods of $\SI{20}{min}$, $\SI{40}{min}$ and $\SI{60}{min}$. In the other hand, $B$ is the amplitude of the trend, set as $\SI{3.84e-6}{TECU.min^{-1}}$ and $t_0=\SI{250}{min}$ determines position of the minimum of the trend.
 
 For the detrending we used a Savitsky-Golay filter of order 3, this is lowest order filter which at the same time avoids too much oscillations and allows to the fit (and its derivative) to be smooth. The other remaining parameter is the window size, i.e. the number of convolution coefficients neccesary for the regression, which must be an odd integer, greater than the order of the polynomial order and lower than the array size. Then, we estimated the detrended signal using all possible values for the window size and estimated the residuals as follows:
 
 \begin{align}
     residuals = \sum_{i}^N \left(d_i - s_i\right)^2
 \end{align}
 Where $N$ is the array size, $d_i$ is the value of the detrended signal at the time $t_i$ and $s_i$ is the reference signal at the same time. In figure  we show the behavior of residuals as a function of the window size relative to the array size. We noted that the residuals behavior is almost insensitive to the window size when its size is lower than than 60\% of the array size, when the errors start to grow exponentially, but going to more detail, we found that using a window size of about $1/4$ of the array length, such errors are minimal. The detrended test curve is shown in figure \ref{fig:detrended-test}, compared with the original signal, as well as the substraction of both curves. In an ideal situation it should be zero for all times, but even in this case the amplitude of the envelope is about the tenth part of the amplitude of the envelope of the signal.

\begin{figure*}
\centering
\includegraphics[width=\linewidth]{../figures/IRvsDetrended}
\label{fig:detrended-test}
\caption{Left: Original test signal (blue continuous curve) compared with the result of detrending the superposition of the test signal with the test trend (figure \ref{fig:test-signal}b), which is the orange dashed curve. Right: Substraction of both curves of left side. The amplitude of the envelope of this residuals qualitatively is about tenth percent of the amplitude of the original signal.}
\end{figure*}

\section{Coordinates transformation}
\label{app:azimuth}

In order to change coordinates form geocentric to local coordinates for the bolide position, we may assume the flat earth approximation since the GPS stations we used in our work are located near the place the bolide was detected, and the apparent curvature of Earth at the bolide height is almost zero. With this in mind, in figure \ref{fig:flat_earth_aproximation} we show the flat Earth approximation in order to transform the geocentric coordinates of the bolide (latitude and longitude) to their corresponding local coordinates azimuth and elevation. With the aid of this image, we derive the following equations:

\begin{align}
    \tan{Az} &= \frac{\Delta \lambda}{\Delta L} \\
    \tan{Ele} &= \frac{h}{r}
\end{align}
Where $r = R_E\left((\Delta L)^2 + (\delta\lambda)^2\right)^{1/2}$ and $(\delta L, \delta\lambda)$ are the separation in latitude and longitude of the bolide from the GPS station, respectively. Azimuth is measured from the north counterclockwise and elevation is measured from the Earth's horizon to zenith, in such way azimuth goes from $0^\circ$ to $360^\circ$ and elevation from $0^\circ$ to $90^\circ$.

\begin{figure}
    \centering
    \includegraphics[width=\linewidth]{../figures/coord_transform_drawing.pdf}
    \caption{Sketch of flat Earth approximation for transforming the fireball position from geocentric coordinates to local coordinates. From this picture we can derive equations () to () to estimate the local azimuth and elevation of the bolide at the GPS station position. $\vec{R}$ is the vector pointing to the bolide position with origin in the GPS station. $r$ is the projection of $\vec{R}$ in the Earth's plane. $L_s$ and $L_B$ are the latitudes of the station and the bolide, respectively, while $\lambda_s$ and $\lambda_B$ their longitudes, $h$ is the bolide height and $R_E$ the radius of Earth.}
    \label{fig:flat_earth_aproximation}
\end{figure}
%To do such transformation we do the following process \citep{Kroger:1996}:

%The transformation from celestial coordinates to local coordinates are given by:

%\begin{align}
%    \sin(E) = \sin{\delta}\sin{L} + \cos{L}\cos{H}\cos\delta \label{eq:elevation}\\
%    \cos{A} = \frac{\sin{\delta}-\sin{E}\sin{L}}{\cos{E}\cos{L}} \label{eq:cos-A}
%\end{align}
%Where $E$ is the elevation of the object, $\delta$ is the declination, $H$ is the hour angle, $A = Az$ when $\sin{H} <0$, otherwise $Az = 360-A$ where $Az$ is the Azimuth. Finally $L$ is the latitude of the observer (i.e, the GPS station). 
%If there were an observer located in the same coordinates than the fragmentation of the bolide took place, then this hypothetical observer should have seen the bolide at zenith. The celestial coordinates of the bolide at this location are zero hour angle and declination equal to object reported latitude. The hour angle is relative to the time and place the object is being observed, and we need the hour angle of the object from the place is observed (the GPS station). To do this we need estimate the hour angle of the object from the station perspective, and the hour angle from the "zenith perspective":
%\begin{align}
%    H = LST_s - RA\label{eq:LST}\\
%    0 = LST_z - RA \label{eq:z-LST}
%\end{align}
%Where $LST_s$ is the Local Sidereal Time from the station perspective, and $LST_z$ is the same but from the "zenith perspective", and $RA$ is the right ascension, which is the same in both perspectives. We can also write the Local sidereal time in terms of the Greenwich local time as follows:%If we sum the last equations we find that:
%\begin{align}
%    LST = GST + Lon_{obs} -RA \label{eq:LST-GST}
%\end{align}

%Where $GST$ is the Greenwich Local time and $Lon_obs$ is the observer longitude. Substituting (\ref{eq:LST-GST}) into (\ref{eq:LST}) and (\ref{eq:z-LST}) and sum the resulting equations we find that:

%\begin{align}
%    H = Lon_s - Lon_z \label{eq:H-lon}
%\end{align}

%Where $Lon_s$ is the station longitude and $Lon_z$ is the longitude of the bolide. Substituting \ref{eq:H-lon} into (\ref{eq:elevation}) and (\ref{eq:cos-A}) we can find the Azimuth and elevation in terms of the local latitude and longitude and the object's reported latitude and longitude:

%\begin{align}
%    \sin(E) = \sin{\delta}\sin{L} + \cos{L}\cos(Lon_s - Lon_z) \label{eq:elevation_f}\\
%    \cos{A} = \frac{\sin{\delta}-\sin{E}\sin{L}}{\cos{E}\cos{L}} \label{eq:cos-A_f}
%\end{align}

%Examples of bolides trajectories against satellites trajectories are found in figures \ref{fig:boav-trajectory} - \ref{fig:ttsf-trajectory} for stations BOAV, CN00, CN04, CN05, KOUR and TTSF:

%\begin{figure}
%    \centering
%    \includegraphics[width=\linewidth]{../figures/azimuth-elevation-map-boav-polar}
%    \caption{Satellites trajectory from BOAV station perspective in colors in polar coordinates, where the azimuth is represented in the polar axis and the elevation in the radial axis, being the zenith at the center of the graph and the horizon at the edge. The estimated meteor trajectory is shown in a black dashed line and GLM data as magenta dots. Practically for all stations, the satellite with PRN 13 is the closest to the bolide trajectory. The dashed curves represent the propagation radius of TIDs after 1 hour (red) and two hours (blue) in the azimuth-elevation space assuming TIDs propagate at \SI{362}{km.s^{-1}}, the same as the reported high frequency TIDs from Chelyabinsk meteor \citep{Yang:2014}.}
%    \label{fig:boav-trajectory}
%\end{figure}

%\begin{figure}
%    \centering
%    \includegraphics[width=\linewidth]{../figures/azimuth-elevation-map-cn00-polar}
%    \caption{Same as figure \ref{fig:boav-trajectory} but using station CN00.}
%    \label{fig:cn00-trajectory}
%\end{figure}\begin{figure}
%    \centering
%    \includegraphics[width=\linewidth]{../figures/azimuth-elevation-map-cn04-polar}
%    \caption{Same as figure \ref{fig:boav-trajectory} but using station CN04.}
%    \label{fig:cn04-trajectory}
%\end{figure}\begin{figure}
%    \centering
%    \includegraphics[width=\linewidth]{../figures/azimuth-elevation-map-cn05-polar}
%    \caption{Same as figure \ref{fig:boav-trajectory} but using station CN05.}
%    \label{fig:cn05-trajectory}
%\end{figure}
%\begin{figure}
%    \centering
%    \includegraphics[width=\linewidth]{../figures/azimuth-elevation-map-kour-polar}
%    \caption{Same as figure \ref{fig:boav-trajectory} but using station KOUR.}
%    \label{fig:kour-trajectory}
%\end{figure}
%\begin{figure}
%    \centering
%    \includegraphics[width=\linewidth]{../figures/azimuth-elevation-map-ttsf-polar}
%    \caption{Same as figure \ref{fig:boav-trajectory} but using station TTSF.}
%    \label{fig:ttsf-trajectory}
%\end{figure}

\section{UNAVCO Acknowledgements and stations list}

In table  we will enlist the stations from we collected data for our work. This material is based on services provided by the GAGE Facility, operated by UNAVCO, Inc., with support from the National Science Foundation and the National Aeronautics and Space Administration under NSF Cooperative Agreement EAR-1724794, we are deeply grateful with all the people that was involved and whose work we are citing here.



\clearpage
\onecolumn
\footnotesize
\begin{landscape}
\begin{table*}
    \centering
       \caption{List of GPS stations used for this work.}
      \label{tab:table-stations}
    \begin{tabular}{lllp{8cm}}
 
    %  \endfirsthead
    %  \endhead
    \toprule
    Station name & Latitude (deg) & Longitude (deg) & Citation \\
    AIRS & 16.74 & -62.21 & \url{https://doi.org/10.7283/T53B5XGJ}\\
    BARA & 18.21 & -71.09 & None available                       \\
    BOAV & 2.85  & -60.70 & None available \\
    CN00 & 17.67 & -61.79 & \url{https://doi.org/10.7283/T5FN14GQ} \\
    CN04 & 14.02 & -60.97 &  \url{https://doi.org/10.7283/T5BP0124} \\
    CN05 & 18.56 & -68.35 & \url{https://doi.org/10.7283/T5VQ30ZH} \\
    CN19 & 12.61 & -70.04 & \url{https://doi.org/10.7283/T5HD7SZB}\\
    CN27 & 19.67 & -69.93 & \url{https://doi.org/10.7283/T5JD4V2P} \\
    CN40 & 12.18 & -68.96 & \url{https://doi.org/10.7283/T5BV7DWT} \\
    CRLR & 18.41 & -68.93 & \url{https://doi.org/10.7283/T5FN14JM} \\
    CRSE & 18.76 & -69.04 & None available \\
    GERD & 16.80 & -62.19 & \url{https://doi.org/10.7283/T5TT4PBT} \\
    GRE1 & 12.22 & -61.64 & \url{https://doi.org/10.7283/T5BC3WZ5} \\
    JME2 & 18.23 & -72.54 &  \url{https://doi.org/10.7283/T5KW5D38}\\
    KOUG & 5.10  & -52.64 & None available \\
    KOUR & 5.25  & -52.81 & None available \\
    LVEG & 19.22 & -70.53 & \url{https://doi.org/10.7283/T5CZ35GC}\\
    NWBL & 16.82 & -62.20 & \url{https://doi.org/10.7283/T5ZK5F13} \\
    OLVN & 16.75 & -62.23 & \url{https://doi.org/10.7283/T5Q23XMD} \\
    RCHY & 16.70 & -62.15 & \url{https://doi.org/10.7283/T5707ZSJ} \\
    RDAZ & 18.45 & -70.72 & None available \\
    RDF2 & 19.45 & -70.68 & None available \\
    RDHI & 18.60 & -68.72 & None available \\
    RDLT & 19.31 & -69.55 & \url{https://doi.org/10.7283/T5J101GT} \\
    RDMA & 19.54 & -71.08 & \url{https://doi.org/10.7283/T50863NM} \\
    RDMC & 19.85 & -71.64 & None available \\
    RDMS & 18.98 & -69.04 & \url{https://doi.org/10.7283/T5DV1HPQ} \\
    RDNE & 18.50 & -71.42 & None available \\
    RDON & 16.93 & -62.35 & \url{https://doi.org/10.7283/T5W37TFB} \\
    RDSD & 18.46 & -69.91 & \url{https://doi.org/10.7283/T5CZ3594} \\
    RDSF & 19.29 & -70.25 & None available \\
    RDSJ & 18.82 & -71.23 & \url{https://doi.org/10.7283/T59W0CTW} \\
    SPED & 18.46 & -69.31 & \url{https://doi.org/10.7283/T5HQ3X75} \\
    SROD & 19.48 & -71.34 & \url{https://doi.org/10.7283/T5862DSD} \\
    TGDR & 18.21 & -71.10 & \url{https://doi.org/10.7283/T5222S3R} \\
    TRNT & 16.76 & -62.16 & \url{https://doi.org/10.7283/T5K935W2} \\
    TTSF & 10.28 & -61.47 & \url{https://doi.org/10.7283/T5JQ0ZCJ} \\
    TTUW & 10.64 & -61.40 & \url{https://doi.org/10.7283/T5TQ5ZTR} \\
    \bottomrule
\end{tabular}     
\end{table*}
  \end{landscape}
  \clearpage
  \twocolumn
\bibliographystyle{model5-names}
\biboptions{authoryear}
\bibliography{../bibliography}

\end{document}

%%

