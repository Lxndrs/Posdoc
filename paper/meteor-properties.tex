\begin{table}
    \centering
     \caption{Caribbean meteor possible physical properties obtained with  algorithm  , assuming a density of \SI{3000}{kg.m^{-3}}}
    \begin{tabular}{lr}\toprule
       Initial Velocity  &  \SI{19.9}{km.s^{-1}}\\
        Initial mass & \SI{340}{kg} \\
        Entry angle (deg) & 64 \\
        \bottomrule
    \end{tabular}
    \label{tab:meteor-properties}
\end{table}

Using the estimation of the meteor energy and velocity from table \ref{tab:Meteor-parameters}, the trajectory angle estimated in appendix \ref{app:velocity} from the velocity components of $64.4^\circ$ and assuming the meteor has a density of \SI{3000}{kg.m^-3}, consistent with the composition of \textit{condrites} (we must assure or edit this last statement), we estimated the initial velocity, initial mass and entry angle using (add citation and more text and once got the references) ...%we are capable to estimate other bolides parameters, such as the mass. For that we used a Fragment Cloud Model \citep{Wheeler:2017}. To obtain the bolide mass, we must have previous knowledge of its velocity. Based on the few events of the USG sample where we could extract the velocity (see table \ref{tab:table-meteors-2}), we tried three models of bolides mass as a function of its releasing energy: the low velocity model, which assumes a velocity of $\sim \SI{15}{km.s{^-1}}$, medium velocty $v \sim \SI{16}{km.s{^-1}}$ and high velocity $v \sim \SI{18}{km.s{^-1}}$. The resultant masses can be seen in figure \ref{fig:energy-mass}. For the rest of the sample we need to assume that the bolides velocity is within this range or close enough in order to get a reasonable estimation of its mass.