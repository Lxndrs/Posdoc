The Earth's magnetic field represents a final obstacle to the Solar Wind (SW) flux. When descelerated and defected by a non colisional shock wave in the flux direction, generates a cavity known as magnetosphere \citep{Blanco-Cano:2004}. Since the Earth is embedded in this SW flux, is known that under adequated physical conditions (e.g magnetic reconnection) may exist some coupling between the magnetosphere and the Earth's ionosphere \citep{Zolesi:2014, Cnossen:2014}.

The Sun plays an important role in the physical proccesses that occur in the terrestrial magnetosphere-ionosphere system. When the SW interacts with the Earth's magnetosphere, particles may permeate the internal region via magnetic reconnection and penetrate to polar zones and generate boreal or austral auroras thus altering the system \citep{Vazquez:2016, Oka:2011}. By the other hand, the Extreme Ultraviolet Radiation (EUV) and X-rays coming from the Sun may interact with the neutral atmoshere via photoionization \citep{Vlasov:2010}. However, in both cases the final result is that the ionosphere's free electrons population is altered.

Some Ionospheric Perturbations (IP) become relevant due to their spatial and temporal scale in the Space Weather scenario. At intermediate latitudes, the most common in the ionosphere are known as Traveling Ionospheric Disturbances (TIDs). Typically they divide into two groups: a) large scale TIDs, associated with geomagnetic storms with sizes of \SI{\sim 2000}{km}, periods of \SI{\sim 1}{h} and velocities of \SI{\sim 700}{km.s^{-1}}, and b) Medium-scale TIDs, which are not fully associated with geomagnetic storms, present sizes of \SI{\sim 100}{km}, periods from 10 minutes to 1 hour and velocities between \SI{50}{km.s^{-1}} and \SI{1e2}{km.s^{-1}} \citep{Helmboldt:2012}. Diverse methods have benn used to study TIDs, such as incoherent dispersion radars, high frequency Doppler emmisors, data from Global Positioning System (GPS) stations or even radiotelescopes like the VLA or the Mexican Array Radio Telescope (MEXART) \citep{Chilcote:2015, Rodriguez:2014}.

On the other side, the Earth's ionosphere may be affected or modified by other proccesses, particularly there are studies that show how the Vetical Total Electron Content (vTEC) due to shock waves generated for rockets launched to space \citep{Lin:2014}. Similar proccesses modify the Earth's ionosphere due to objects entering the athmosphere from space, such as meteoroids like the one which fell on Chelyabinsk at 2013 \citep{Yang:2014}. Previously, the ionospheric perturbations produced by this object were studied using two independent methods: a) detecting vTEC pertubations using GPS station near the impact location. And b) a wavelets analysis for detection of ...

In 2020 a meteoroid passed in mexican territory through mexican territory, which also was studied \citep{Sergeeva:2020}. The meteoroid was recorded with outdoor cameras in different locations. The trajectory could be estimated, as well as other physical parameters.

In this work we will show a similar analysis for a sample of meteoroids detected in mexican territory by different methods. The first subsample consists in objects detected by the Geostationary Lightning Mapper (GLM) whose sizes are estimated between a few decimeters to meters in diameter \citep{GOODMAN:2013, Jenniskens:2018, Rumpf:2019}. The second subsample will consist in objects detected by ocular witnesses from the American Meteor Society and as comparisson we will include the morelian meteoroid reported in \citet{Sergeeva:2020} and the Chelyabinsk event \citet{Yang:2014}. The paper is arranged in the following way: \S \ref{sec:methodology} describes the samples of meteoroids as well of the properties that can be obtained from direct observations. Also describes the GPS data corresponding to the dates and locations where each object was located. \S \ref{sec:bolides} shows physical parameters of meteoroids obtained from the observed heights and energies. Finally, section \S \ref{sec:vTEC-maps} shows the vTEC maps and scintillation indices obtained from GPS observations.  