Bolides impact the Earth every day, from the smallest ($d<\SI{0.5}{m}$), which are destroyed when entering the atmosphere to the biggest and most rare events, where an asteroid with a diameter $d>\SI{100}{m}$ is almost unaffected by the atmosphere and is technically unable to slow them down. If their diameter is greater than \SI{1}{km} the impact can be considered a global catastrophe \citep{Dudorov:2020}. Some famous modern cases are the Tunguska event 1n 1908 \citep{Wheeler:2019} and the Chelyabinsk event \citep{Yang:2014} in 2013.

Such events have been recently recorded in certain databases, such as the Center for Near Earth Objects Studies (CNEOS), Jet Propulsion Laboratory (JPL) and the Geostationary Lightning Mapper (GLM), publicly available at \url{https://cneos.jpl.nasa.gov/fireballs/} and \url{https://neo-bolide.ndc.nasa.gov/#/}, respectively.

We picked our interest in one certain event, that we nicknamed ``the Caribbean Meteoroid'', because this meteoroid is the most energetic which appear in both databases and thus is an interesting case for studying the potential Traveling Ionospheric Disturbances (TIDs) that could be produced. To do so, we collected RINEX data from nearby GPS stations from the UNAVCO network (also publicly available) and developed a method to detrend the resultant TEC series and estimate the TIDs propagation speed ... (The introduction also needs to be polished after getting the main results)

