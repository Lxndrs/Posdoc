We collected data for a sample of GPS station from UNAVCO network around the Caribbean bolide was detected in the same day and next day. We used data from the GLM, where the total energy was estimated from light curves, and USG sensors, where energy data is available to spot the bolide position and estimate the trajectory at the fragmentation instant. 
For GPS data, we adapted the method from \citet{Pradipta:2015} for detrending all the resultant TEC curves to remove the effects of Earth rotation and solar activity and make the wave-like features more evident. In order to discriminate which stations made a detection from the ones who didn't, we compared the detrended time series with a Morlet wavelet. We discarded other sources of GWs such like slow solar wind reached by fast wind, fluctuations in solar X-rays, coronal mass ejections and solar terminator effects. We found TIDs features in sTEC series from stations CN04, CN40, GRE1, TTSF, TTUW, BOAV, KOUG and KOUR stations with at least one satellite. 
