We collected data for a sample of meteors which were detected around Mexican territory, or with some exceptions, meteoroids that released enough energy to have a good chance to get a vTEC disturbance with nearby GPS stations. We used data from the GLM, where the total energy was estimated from light curves, and USG sensors, where energy data is available. We used the few events which are present in both samples to re-calibrate the energy obtained for the GLM sample since in this few cases the energy estimated with the GLM data is systematically lower than the energy reported by USG sensors. Using the resulting energies, we estimated the bolides mass using a Fragment Cloud Model assuming a certain velocity range. 

We have collected GPS data of nearby stations where the event was detected at the date each event occured, and one day before as comparison. We obtained the the vTEC series and we detrended them to remove the effects of Earth rotation and solar activity and make the wave-like features more evident. With these data, we used a wavelet transform using a Morlet wavelet to discover the frequencies at which vTEC perturbations may be produced, how intense these perturbations are and the time interval of these perturbations. If two different stations detect perturbations at the same frequencies that means that a vTEC perturbation was detected with such stations. We have observed that the distribution of wavelet coherent signal show us that the propagation of ionospheric disturbances have a time intervals of occurrence of a two hours from the spectrograms. 
