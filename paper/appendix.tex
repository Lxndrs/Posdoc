\appendix
% \appendixpage
% \addappheadtotoc

% This is the appendix, where we may insert detailed mathematical procedures or other data which may be useful but also will make the main
% text too large so this should be consulted only in case of necessity
\section{Bolide velocity components}
\label{app:velocity}
The USG database from JPL is a great source of information about the brightest bolides that have entered the atmosphere since 1988. However, the velocity components $(v_x, v_y, v_z)$ are displayed in a geocentric Earth-fixed reference frame explained as follows:

$v_x$ lies in the Earth's equatorial plane, parallel to the equator and points towards the prime meridian, $v_z$ is parallel to the earth's rotational axis and points towards the north celestial pole and $v_y$ completes the right-handed reference system. Figure  shows graphically these velocity components.

In the other hand, it might be useful and more intuitive to describe the velocity components in terms of the local reference system composed by $(v_{lon}, v_{lat}, v_h)$, where $v_{lon}$ is the velocity component parallel to the equatorial plane, positive towards east, $v_{lat}$ is parallel to the polar plane, positive to north, and $v_h$ is the radial component pointing towards the center of Earth. Figure \ref{fig:ref_velocity} sketches both reference systems simultaneously.

\begin{figure}
    \centering
    \includegraphics[width=\linewidth]{../figures/meteor_velocity_references.pdf}
    \caption{Velocity components of bolide. The geocentric Earth fixed reference system is sketched in black and the local spherical reference system in blue. The components parallel to the equatorial plane are not shown since are perpendicular to the plane of the page.}
    \label{fig:ref_velocity}
\end{figure}

The transformation between both reference systems in the north-west hemisphere is given as follows:

\begin{align}
v_{lon} &= v_x \label{eq:vlon}\\
v_{lat} &= v_z\cos{L} - v_y\sin{L} \label{eq:vlat}\\
v_h     &= v_y\cos{L} + v_z\sin{L} \label{eq:vh}
\end{align}

Where $L$ is the bolide latitude. With this information, we can estimate the velocity tangential to the Earth's surface as $v_{tan} = \left((v_{lon})^2 + (v_{lat})^2\right)^{1/2}$ and finally the trajectory angle respect to the normal as $\tan\theta = \frac{v_{tan}}{v_h}$. Using equations (\ref{eq:vlon}) to (\ref{eq:vh}) we obtain the following:

\begin{align}
    v_{tan} &= \left((v_x)^2 + (v_z\cos{L}-v_y\sin{L})^2\right)^{1/2} \\
    \tan\theta &= \frac{\left((v_x)^2 + (v_z\cos{L}-v_y\sin{L})^2\right)^{1/2}}{v_y\cos{L} + v_z\sin{L}}  \label{eq:meteor-angle}
\end{align}

Substituting the corresponding data from table \ref{tab:Meteor-parameters} into equation (\ref{eq:meteor-angle}) we find that that the trajectory angle of the meteor respect to the normal is $\theta \approx 64.4^{\circ}$

\section{Detrending test signal}
\label{app:test-signal}
Figure \ref{fig:test-signal} shows the form of this test time series, described mathematically as follows \citep{Boris:2020}:
 \begin{align}
     I_R(t) &= A\cdot \exp\left[-0.5\left(\frac{t-t_m}{d_t}\right)^2\right]\cdot \sum^{i}_{n}\sin\left(\omega_i t\right) \label{eq:ref-signal}\\
     Trend(t) &= B\cdot \left|t - t_0\right|^3 \label{eq:trend}
 \end{align}
 
 \begin{figure*}
     \centering
     \includegraphics[width=\linewidth]{../figures/test_signal}
     \caption{Initial setup for testing detrending method. a) Reference signal given by equation (\ref{eq:ref-signal}). b) Superposition between a) and the trend given by (\ref{eq:trend})}.
     \label{fig:test-signal}
 \end{figure*}
 
 Where equation (\ref{eq:ref-signal}) is the signal we want to detrend and equation (\ref{eq:trend}) is the trend we want to remove. $A$ is the amplitude of the signal, set as $\SI{0.2}{TECU}$, $t_m$ is a parameter which determines the position of the envelope maximum, set as $\SI{250}{min}$, which is the half of the array length, $d_t$ is the half width of the envelope, set as $\SI{50}{min}$ and $\omega_i$ are the frequencies of three harmonics with periods of $\SI{20}{min}$, $\SI{40}{min}$ and $\SI{60}{min}$. In the other hand, $B$ is the amplitude of the trend, set as $\SI{3.84e-6}{TECU.min^{-1}}$ and $t_0=\SI{250}{min}$ determines position of the minimum of the trend.
 
 For the detrending we used a Savitsky-Golay filter of order 3, this is lowest order filter which at the same time avoids too much oscillations and allows to the fit (and its derivative) to be smooth. The other remaining parameter is the window size, i.e. the number of convolution coefficients neccesary for the regression, which must be an odd integer, greater than the order of the polynomial order and lower than the array size. Then, we estimated the detrended signal using all possible values for the window size and estimated the residuals as follows:
 
 \begin{align}
     residuals = \sum_{i}^N \left(d_i - s_i\right)^2
 \end{align}
 Where $N$ is the array size, $d_i$ is the value of the detrended signal at the time $t_i$ and $s_i$ is the reference signal at the same time. In figure  we show the behavior of residuals as a function of the window size relative to the array size. We noted that the residuals behavior is almost insensitive to the window size when its size is lower than than 60\% of the array size, when the errors start to grow exponentially, but going to more detail, we found that using a window size of about $1/4$ of the array length, such errors are minimal. The detrended test curve is shown in figure \ref{fig:detrended-test}, compared with the original signal, as well as the substraction of both curves. In an ideal situation it should be zero for all times, but even in this case the amplitude of the envelope is about the tenth part of the amplitude of the envelope of the signal.

\begin{figure*}
\centering
\includegraphics[width=\linewidth]{../figures/IRvsDetrended}
\label{fig:detrended-test}
\caption{Left: Original test signal (blue continuous curve) compared with the result of detrending the superposition of the test signal with the test trend (figure \ref{fig:test-signal}b), which is the orange dashed curve. Right: Substraction of both curves of left side. The amplitude of the envelope of this residuals qualitatively is about tenth percent of the amplitude of the original signal.}
\end{figure*}

\section{Coordinates transformation}
\label{app:azimuth}

In order to change coordinates form geocentric to local coordinates for the bolide position, we may assume the flat earth approximation since the GPS stations we used in our work are located near the place the bolide was detected, and the apparent curvature of Earth at the bolide height is almost zero. With this in mind, in figure \ref{fig:flat_earth_aproximation} we show the flat Earth approximation in order to transform the geocentric coordinates of the bolide (latitude and longitude) to their corresponding local coordinates azimuth and elevation. With the aid of this image, we derive the following equations:

\begin{align}
    \tan{Az} &= \frac{\Delta \lambda}{\Delta L} \\
    \tan{Ele} &= \frac{h}{r}
\end{align}
Where $r = R_E\left((\Delta L)^2 + (\delta\lambda)^2\right)^{1/2}$ and $(\delta L, \delta\lambda)$ are the separation in latitude and longitude of the bolide from the GPS station, respectively. Azimuth is measured from the north counterclockwise and elevation is measured from the Earth's horizon to zenith, in such way azimuth goes from $0^\circ$ to $360^\circ$ and elevation from $0^\circ$ to $90^\circ$.

\begin{figure}
    \centering
    \includegraphics[width=\linewidth]{../figures/coord_transform_drawing.pdf}
    \caption{Sketch of flat Earth approximation for transforming the fireball position from geocentric coordinates to local coordinates. From this picture we can derive equations () to () to estimate the local azimuth and elevation of the bolide at the GPS station position. $\vec{R}$ is the vector pointing to the bolide position with origin in the GPS station. $r$ is the projection of $\vec{R}$ in the Earth's plane. $L_s$ and $L_B$ are the latitudes of the station and the bolide, respectively, while $\lambda_s$ and $\lambda_B$ their longitudes, $h$ is the bolide height and $R_E$ the radius of Earth.}
    \label{fig:flat_earth_aproximation}
\end{figure}
%To do such transformation we do the following process \citep{Kroger:1996}:

%The transformation from celestial coordinates to local coordinates are given by:

%\begin{align}
%    \sin(E) = \sin{\delta}\sin{L} + \cos{L}\cos{H}\cos\delta \label{eq:elevation}\\
%    \cos{A} = \frac{\sin{\delta}-\sin{E}\sin{L}}{\cos{E}\cos{L}} \label{eq:cos-A}
%\end{align}
%Where $E$ is the elevation of the object, $\delta$ is the declination, $H$ is the hour angle, $A = Az$ when $\sin{H} <0$, otherwise $Az = 360-A$ where $Az$ is the Azimuth. Finally $L$ is the latitude of the observer (i.e, the GPS station). 
%If there were an observer located in the same coordinates than the fragmentation of the bolide took place, then this hypothetical observer should have seen the bolide at zenith. The celestial coordinates of the bolide at this location are zero hour angle and declination equal to object reported latitude. The hour angle is relative to the time and place the object is being observed, and we need the hour angle of the object from the place is observed (the GPS station). To do this we need estimate the hour angle of the object from the station perspective, and the hour angle from the "zenith perspective":
%\begin{align}
%    H = LST_s - RA\label{eq:LST}\\
%    0 = LST_z - RA \label{eq:z-LST}
%\end{align}
%Where $LST_s$ is the Local Sidereal Time from the station perspective, and $LST_z$ is the same but from the "zenith perspective", and $RA$ is the right ascension, which is the same in both perspectives. We can also write the Local sidereal time in terms of the Greenwich local time as follows:%If we sum the last equations we find that:
%\begin{align}
%    LST = GST + Lon_{obs} -RA \label{eq:LST-GST}
%\end{align}

%Where $GST$ is the Greenwich Local time and $Lon_obs$ is the observer longitude. Substituting (\ref{eq:LST-GST}) into (\ref{eq:LST}) and (\ref{eq:z-LST}) and sum the resulting equations we find that:

%\begin{align}
%    H = Lon_s - Lon_z \label{eq:H-lon}
%\end{align}

%Where $Lon_s$ is the station longitude and $Lon_z$ is the longitude of the bolide. Substituting \ref{eq:H-lon} into (\ref{eq:elevation}) and (\ref{eq:cos-A}) we can find the Azimuth and elevation in terms of the local latitude and longitude and the object's reported latitude and longitude:

%\begin{align}
%    \sin(E) = \sin{\delta}\sin{L} + \cos{L}\cos(Lon_s - Lon_z) \label{eq:elevation_f}\\
%    \cos{A} = \frac{\sin{\delta}-\sin{E}\sin{L}}{\cos{E}\cos{L}} \label{eq:cos-A_f}
%\end{align}

%Examples of bolides trajectories against satellites trajectories are found in figures \ref{fig:boav-trajectory} - \ref{fig:ttsf-trajectory} for stations BOAV, CN00, CN04, CN05, KOUR and TTSF:

%\begin{figure}
%    \centering
%    \includegraphics[width=\linewidth]{../figures/azimuth-elevation-map-boav-polar}
%    \caption{Satellites trajectory from BOAV station perspective in colors in polar coordinates, where the azimuth is represented in the polar axis and the elevation in the radial axis, being the zenith at the center of the graph and the horizon at the edge. The estimated meteor trajectory is shown in a black dashed line and GLM data as magenta dots. Practically for all stations, the satellite with PRN 13 is the closest to the bolide trajectory. The dashed curves represent the propagation radius of TIDs after 1 hour (red) and two hours (blue) in the azimuth-elevation space assuming TIDs propagate at \SI{362}{km.s^{-1}}, the same as the reported high frequency TIDs from Chelyabinsk meteor \citep{Yang:2014}.}
%    \label{fig:boav-trajectory}
%\end{figure}

%\begin{figure}
%    \centering
%    \includegraphics[width=\linewidth]{../figures/azimuth-elevation-map-cn00-polar}
%    \caption{Same as figure \ref{fig:boav-trajectory} but using station CN00.}
%    \label{fig:cn00-trajectory}
%\end{figure}\begin{figure}
%    \centering
%    \includegraphics[width=\linewidth]{../figures/azimuth-elevation-map-cn04-polar}
%    \caption{Same as figure \ref{fig:boav-trajectory} but using station CN04.}
%    \label{fig:cn04-trajectory}
%\end{figure}\begin{figure}
%    \centering
%    \includegraphics[width=\linewidth]{../figures/azimuth-elevation-map-cn05-polar}
%    \caption{Same as figure \ref{fig:boav-trajectory} but using station CN05.}
%    \label{fig:cn05-trajectory}
%\end{figure}
%\begin{figure}
%    \centering
%    \includegraphics[width=\linewidth]{../figures/azimuth-elevation-map-kour-polar}
%    \caption{Same as figure \ref{fig:boav-trajectory} but using station KOUR.}
%    \label{fig:kour-trajectory}
%\end{figure}
%\begin{figure}
%    \centering
%    \includegraphics[width=\linewidth]{../figures/azimuth-elevation-map-ttsf-polar}
%    \caption{Same as figure \ref{fig:boav-trajectory} but using station TTSF.}
%    \label{fig:ttsf-trajectory}
%\end{figure}

\section{UNAVCO Acknowledgements and stations list}

In table  we will enlist the stations from we collected data for our work. This material is based on services provided by the GAGE Facility, operated by UNAVCO, Inc., with support from the National Science Foundation and the National Aeronautics and Space Administration under NSF Cooperative Agreement EAR-1724794, we are deeply grateful with all the people that was involved and whose work we are citing here.



\clearpage
\onecolumn
\footnotesize
\begin{landscape}
\begin{table*}
    \centering
       \caption{List of GPS stations used for this work.}
      \label{tab:table-stations}
    \begin{tabular}{lllp{8cm}}
 
    %  \endfirsthead
    %  \endhead
    \toprule
    Station name & Latitude (deg) & Longitude (deg) & Citation \\
    AIRS & 16.74 & -62.21 & \url{https://doi.org/10.7283/T53B5XGJ}\\
    BARA & 18.21 & -71.09 & None available                       \\
    BOAV & 2.85  & -60.70 & None available \\
    CN00 & 17.67 & -61.79 & \url{https://doi.org/10.7283/T5FN14GQ} \\
    CN04 & 14.02 & -60.97 &  \url{https://doi.org/10.7283/T5BP0124} \\
    CN05 & 18.56 & -68.35 & \url{https://doi.org/10.7283/T5VQ30ZH} \\
    CN19 & 12.61 & -70.04 & \url{https://doi.org/10.7283/T5HD7SZB}\\
    CN27 & 19.67 & -69.93 & \url{https://doi.org/10.7283/T5JD4V2P} \\
    CN40 & 12.18 & -68.96 & \url{https://doi.org/10.7283/T5BV7DWT} \\
    CRLR & 18.41 & -68.93 & \url{https://doi.org/10.7283/T5FN14JM} \\
    CRSE & 18.76 & -69.04 & None available \\
    GERD & 16.80 & -62.19 & \url{https://doi.org/10.7283/T5TT4PBT} \\
    GRE1 & 12.22 & -61.64 & \url{https://doi.org/10.7283/T5BC3WZ5} \\
    JME2 & 18.23 & -72.54 &  \url{https://doi.org/10.7283/T5KW5D38}\\
    KOUG & 5.10  & -52.64 & None available \\
    KOUR & 5.25  & -52.81 & None available \\
    LVEG & 19.22 & -70.53 & \url{https://doi.org/10.7283/T5CZ35GC}\\
    NWBL & 16.82 & -62.20 & \url{https://doi.org/10.7283/T5ZK5F13} \\
    OLVN & 16.75 & -62.23 & \url{https://doi.org/10.7283/T5Q23XMD} \\
    RCHY & 16.70 & -62.15 & \url{https://doi.org/10.7283/T5707ZSJ} \\
    RDAZ & 18.45 & -70.72 & None available \\
    RDF2 & 19.45 & -70.68 & None available \\
    RDHI & 18.60 & -68.72 & None available \\
    RDLT & 19.31 & -69.55 & \url{https://doi.org/10.7283/T5J101GT} \\
    RDMA & 19.54 & -71.08 & \url{https://doi.org/10.7283/T50863NM} \\
    RDMC & 19.85 & -71.64 & None available \\
    RDMS & 18.98 & -69.04 & \url{https://doi.org/10.7283/T5DV1HPQ} \\
    RDNE & 18.50 & -71.42 & None available \\
    RDON & 16.93 & -62.35 & \url{https://doi.org/10.7283/T5W37TFB} \\
    RDSD & 18.46 & -69.91 & \url{https://doi.org/10.7283/T5CZ3594} \\
    RDSF & 19.29 & -70.25 & None available \\
    RDSJ & 18.82 & -71.23 & \url{https://doi.org/10.7283/T59W0CTW} \\
    SPED & 18.46 & -69.31 & \url{https://doi.org/10.7283/T5HQ3X75} \\
    SROD & 19.48 & -71.34 & \url{https://doi.org/10.7283/T5862DSD} \\
    TGDR & 18.21 & -71.10 & \url{https://doi.org/10.7283/T5222S3R} \\
    TRNT & 16.76 & -62.16 & \url{https://doi.org/10.7283/T5K935W2} \\
    TTSF & 10.28 & -61.47 & \url{https://doi.org/10.7283/T5JQ0ZCJ} \\
    TTUW & 10.64 & -61.40 & \url{https://doi.org/10.7283/T5TQ5ZTR} \\
    \bottomrule
\end{tabular}     
\end{table*}
  \end{landscape}
  \clearpage
  \twocolumn